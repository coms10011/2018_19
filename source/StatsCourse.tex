\documentclass[12pt,a4paper]{scrartcl}
\typearea{10}


\usepackage{graphicx}
\usepackage{amsfonts}
\usepackage{amssymb}
\usepackage{amsmath}
\usepackage{amsthm}


\begin{document}

\section*{Probability and Statistics}
This course will teach and understanding of probability and the skills
needed to analyse data.

\subsection*{Overview}

Computer science is the science of computation and data. Our approach
to data is typically very sophisticated and algorithm driven, however,
it is often different from the approaches used in many of the other
disciplines to which computer scientists contribute. In many areas of
science and in enterprise, data is analysed using a frequentist
approach which has been developed over the last two centuries into a
powerful and practical method for understanding and interpreting data,
particularly the data that results from experiments, whether the
empirical results of scientific experiments or data collected in
pursuit of business goals. The aim of this course is to introduce
computer science students to these methods and to teach them
statistical skills which are both useful as an approach to data and as
the language of data commonly employed in science and industry.

\subsection*{Aims}

The aim of this course is to introduce students to mathematics
underpinning statistics, to the methods commonly used to analyse data.

\subsection*{Learning outcomes}

At the end of this course the students will:
\begin{itemize}
\item Understand the foundations of probability and statistics.
\item Be familiar with sampling and sampling bias.
\item Know and be able interpret different distributions.
\item Make the best use of descriptive statistics.
\item Be able to analyse data using classical statistical tests.
\item Understand the design and analyse of experiments.
\end{itemize}

\subsection*{Syllabus}
\begin{itemize}
\item Random variables, probabilities, probability mass functions and Bayes rule.
\item Counting and selection; the binomial distribution.
\item Moments of a distribution include the mean, variance; sampling and bias.
\item Poisson processes, the Laplace and gamma distribution.
\item The law of large numbers. The central limit theorem: the
  Gaussian distribution.
\item Simulating samples from specified distributions. 
\item Hypothesis testing. The Student t-test.
\item Test of normality.
\item Summary statistics: the Wilcoxon signed rank test and Wilcoxon-Mann-Whitney U test.
\item Regression, slope and intercept, the statistical significance of regression.
\item Understanding basic design principles, one-way ANOVA, the Kruskal-Wallis and Friedman tests.

\end{itemize}

\subsection*{Teaching and assessment}
The course will include lectures and practical workshops, the
workshops will included marked coursework worth 10\% of the final
mark. A 2 hour exam will make up the remaining 90\%.

\end{document}
