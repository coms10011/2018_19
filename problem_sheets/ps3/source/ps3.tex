%ps3.tex
%notes for the course Probability and Statistics COMS10011 
%taught at the University of Bristol
%2018_19 Conor Houghton conor.houghton@bristol.ac.uk

%To the extent possible under law, the author has dedicated all copyright 
%and related and neighboring rights to these notes to the public domain 
%worldwide. These notes are distributed without any warranty. 

\documentclass[11pt,a4paper]{scrartcl}
\typearea{12}
\usepackage{graphicx}
%\usepackage{pstricks}
\usepackage{listings}
\usepackage{color}
\lstset{language=C}
\usepackage{fancyhdr}
\pagestyle{fancy}
\lfoot{\texttt{github.com/COMS10011/2018\_19}}
\lhead{COMS10007 ps3 - Conor}
\begin{document}

\section*{Problem Sheet 3}

There are two questions from Conor and two from Anne; Anne's question come after the \lq{}additional problem\rq{} from Conor's part of the problem sheet.

\subsection*{Useful facts}

\begin{itemize}

\item \textbf{The Gau\ss{}ian distribution}:
$$
p(x)=\frac{1}{\sqrt{2\pi\sigma^2}}e^{-\frac{(x-\mu)^2}{2\sigma^2}}
$$

\item Working out \textbf{probabilities for the Gau\ss{}ian}:
$$
\mbox{Prob}(x_1<x<x_2)=\frac{1}{2}[\mbox{erf}\,(z_2)-\mbox{erf}\,(z_1)]
$$
where
$$
z=\frac{x-\mu}{\sqrt{2}\sigma}
$$
The error function 
$$\mbox{erf}\,(z)=\frac{1}{\sqrt{\pi}}\int_{-z}^ze^{-y^2}dy=\frac{2}{\sqrt{\pi}}\int_0^ze^{-y^2}dy$$ 
goes from -1 to one, so $\mbox{erf}\,(-\infty)=-1$ and $\mbox{erf}\,(\infty)=1$.
\end{itemize}



\subsection*{Questions}

\begin{enumerate}

\item The size of a standard croquet ball is 3 5/8
  inches\footnote{Everything in croquet is measured in old timey
    units}. The height of a croquet hoop is 3 3/4 inches. If a not
  very good croquet-ball making machine makes croquet balls whose mean
  matches the standard and with standard deviation 1/8 inch, what is
  the chance it will make a ball too large to fit through the hoop?
  You can write the solution in terms of the error function.



\item This will look like a long question but it is almost all
  background and the question is not too bad when you actually read
  through it. In particle physics when a collider is being used to
  find a new particle like the Higgs boson or the top squark
  scientists don't detect the sought after particle directly since it
  usually decays almost straight away, instead they detect the more
  common particles that particle will decay into, for example, a Higgs
  boson can decay in to two photons and these can be detected. Roughly
  speaking scientists count these events. However, the whole situation
  is very messy and there will always be some events even if the
  particle doesn't exist at the energy being examined. The amount of
  these background events will fluctuate from experiment to
  experiment, typically like a Gau\ss{}ian. The scientific team is
  allowed to claim they have discovered the particle if the number of
  events they measure is more than five standard deviations above
  what would be expected if the particle didn't exist. What is the
  probability of this \lq{}discovery\rq{} happening by chance?



\end{enumerate}

\subsection*{Extra questions}

These are for you to do on your own, not for handing up. Solutions will be included in the solutions section. I haven't added these question yet, but they will be added to the online version of this problem sheet over the next couple of days.

\end{document}

