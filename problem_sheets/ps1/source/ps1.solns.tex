%ps1.solns.tex
%notes for the course Probability and Statistics COMS10011 
%taught at the University of Bristol
%2018_19 Conor Houghton conor.houghton@bristol.ac.uk

%To the extent possible under law, the author has dedicated all copyright 
%and related and neighboring rights to these notes to the public domain 
%worldwide. These notes are distributed without any warranty. 

\documentclass[11pt,a4paper]{scrartcl}
\typearea{12}
\usepackage{graphicx}
%\usepackage{pstricks}
\usepackage{listings}
\usepackage{color}
\lstset{language=C}
\usepackage{fancyhdr}
\pagestyle{fancy}
\lfoot{\texttt{github.com/COMS10011/2018\_19}}
\lhead{COMS10007 ps1.solns - Conor}
\begin{document}

\section*{Problem Sheet 1 - outline solutiosn}



\begin{enumerate}



\item In the poker hand two pair there are two pairs of cards with
  each card in the pair matched by value; the fifth card is a
  different value. What is the probability of two pairs when five
  cards are drawn randomly. In a full house there is one pair and one
  triple, what is the probability of getting a full
  house?\\ \\ \\ \textbf{Solution}: There are 13 choose two choices
  for the two values for the two pairs and for each pair there are
  four choose two possible cards. For the remaining card there are 11
  possible values and four possible suits. Thus, the number of
  possible pairs is
\begin{equation}
\left(\begin{array}{c}13\\2\end{array}\right)\left(\begin{array}{c}4\\2\end{array}\right)^2\times 44
=
\frac{13\times 12}{1\times 2}\times 36\times 44=123552
\end{equation}
and hence the probability is 123552/2598960=0.0475. For full house,
there are 13 possible values for the pair and 12 for the triple; including the choice of suits we have
\begin{equation}
13\times 12 \times \left(\begin{array}{c}4\\2\end{array}\right)\left(\begin{array}{c}4\\3\end{array}\right)=3744
\end{equation}
and the probability is 0.0014.

\item A student answers a multiple choice question with four options,
  one of which is correct. 80\% of students know the answer, 20\% of
  students guess and choose randomly. If a student gets the answer
  correct what is the chance they knew the
  answer.\\ \\ \\ \textbf{Solution}: Let $K$ be the event the student
  knows the right answer and $C$ is the event that the student chooses
  the correct answer. We want $P(K|C)$. By Bayes's rule
\begin{equation}
P(K|C)=\frac{P(C|K)P(K)}{P(C)}
\end{equation}
Now
\begin{equation}
  P(C)=P(C|K)P(K)+P(C|\bar{K})P(\bar{K})=0.8+0.25*0.2=0.85
\end{equation}
and hence
\begin{equation}
P(K|C)=\frac{0.8}{0.85}=0.94
\end{equation}

\item In the xkcd cartoon above, what is the chance the Bayesian will
  when his or her bet if the chance the sun has exploded is one in a
  million? In reality the chance is, of course, much less than one in
  a million! Show the answer to six decimal places.\\ \\ \\ \textbf{Solution}: Let $N$ be the event that the sun has exploded
  and $L$ be the event the machine says that the sun has
  exploded. Hence $P(L|N)=35/36$ whereas $P(L|\bar{N})=1/36$ and $P(N)=10^{-6}$. The Bayesian will
  win his or her bet is $P(\bar{N}|L)$:
\begin{equation}
P(\bar{N}|L)=\frac{P(L|\bar{N})P(\bar{N})}{P(L)}=\frac{1/36\times (1-10^{-6})}{1/36\times (1-10^{-6})+35/36\times 10^{-6}}=0.999965
\end{equation}


\item A three-sided dice is rolled three times. $X$ is the sum of the
  largest two values. Write down the probability distribution for
  $X$.\\ \\ \\ \textbf{Solution}: Well lets write down table and then
  explain where we got the numbers from
\begin{center}
\begin{tabular}{c|ccccc}
&2&3&4&5&6\\
\hline
$p_X$&1/27&1/9&7/27&1/3&7/27
\end{tabular}
\end{center}
So there are 27 possible outcomes of rolling the dice three times. To
get $X=2$ you need to roll 111, to get $X=3$ you can roll 112, 121 or
211. To get $X=4$ there are three permutations of 113 and three
permutations of 122, along with 222. To get $X=5$ there are six
permutations of 123 along with three permutations of 223. Finally to
rolls six there are three permutations of each of 133 and 233, along
with 333.

\end{enumerate}

\end{document}

