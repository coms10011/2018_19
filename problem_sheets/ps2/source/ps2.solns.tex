%ps2.tex
%notes for the course Probability and Statistics COMS10011 
%taught at the University of Bristol
%2018_19 Conor Houghton conor.houghton@bristol.ac.uk

%To the extent possible under law, the author has dedicated all copyright 
%and related and neighboring rights to these notes to the public domain 
%worldwide. These notes are distributed without any warranty. 

\documentclass[11pt,a4paper]{scrartcl}
\typearea{12}
\usepackage{graphicx}
%\usepackage{pstricks}
\usepackage{listings}
\usepackage{color}
\lstset{language=C}
\usepackage{fancyhdr}
\pagestyle{fancy}
\lfoot{\texttt{github.com/COMS10011/2018\_19}}
\lhead{COMS10007 ps2.solns - Conor}
\begin{document}

\section*{Problem Sheet 2 - outline solutions}

\subsection*{Questions}

Four questions, each worth two marks with two marks for attendance.

\begin{enumerate}


\item The illusionist Derren Brown famously flipped a coin on camera
  so that it landed heads ten times in a row; he claimed that this was
  because of his mind powers, in fact it was because of his patience,
  he simply kept trying the trick again and again until it
  worked. It took him nine hours. What is the probability of a coin landing heads ten times in
  a row? If you flip a coin ten times what is the probability of
  getting five heads and five tails?\\ \\ \\ \textbf{Solution}: The chance of 10 heads is
\begin{equation}
p(10)=0.5^{10}=0.0009765625
\end{equation}
whereas the chance of five heads is
\begin{equation}
p(5)=\left(\begin{array}{c}10\\5\end{array}\right)0.5^{10}=0.24609375
\end{equation}


\item A fisher catches on average a fish every 25 minutes. What is the
  probability that they catche no fish in an
  hour?\\ \\ \\ \textbf{Solution}: If he catches a fish every 25
  minutes then his rate for an hour is 60/25=2.4 so
\begin{equation}
p(0)=e^{-2.4}\approx 0.09
\end{equation}

\item The distribution of tree heights in a christmas tree forest is 
\begin{equation}
p(h)=\left\{\begin{array}{cc} 0.3& 0\le h <2\\0.2& 2\le h<4\\0&\mbox{otherwise}\end{array}\right.
\end{equation}
What is the mean height of trees in the forest?
\\ \\ \\
\textbf{Solution}: So 
\begin{equation}
\mu =\langle H\rangle=\int_{-\infty}^\infty hf(h)=\int_{0}^2 {0.3h}dh+\int_2^4{0.2h}dh
\end{equation}
and
\begin{equation}
\int_{0}^2 {0.3h}dh=\left.0.3\frac{h^2}{2}\right|_0^2=0.6
\end{equation}
and
\begin{equation}
\int_{2}^4 {0.2h}dh=\left.0.2\frac{h^2}{2}\right|_2^4=0.2 (8-2)=1.2
\end{equation}
so $\mu=1.8$.

\item Like the binomial distribution the geometric probability
  distribution is related to a series of independent trials where each
  trial has probability $p$ of success and $q=1-p$ of failure. The
  geometric probability $p(r)$ is the probability that the $r$th trial
  is the first success. It is
\begin{equation}
p(r)=q^{r-1}p
\end{equation}
It can be shown that 
\begin{equation}
\sum_{r=1}^\infty p(r)=1
\end{equation}
as it must be. You can assume that here. What is the mean of the geometric probability? As a hint, this is done much as for the binomial expansion.
\\ \\ \\ 
\textbf{Solution}: So we have
\begin{equation}
1=\sum_{r=1}^\infty q^{r-1}p
\end{equation}
If we differenciate both sides by $p$ we get
\begin{equation}
0=\sum_{r=1}^\infty q^{r-1}-\sum_{r=1}^\infty (r-1)q^{r-2}p
\end{equation}
or
\begin{equation}
0=\frac{1}{p}\sum_{r=1}^\infty q^{r-1}p-\sum_{r=1}^\infty (r-1)q^{r-2}p
\end{equation}
In the second term set $s=r-1$ to get 
\begin{equation}
\sum_{r=1}^\infty (r-1)q^{r-2}p=\sum_{s=0}^\infty sq^{s-1}p=\sum_{s=1}^\infty sq^{s-1}p
\end{equation}
where we are able to change drop the $s=0$ term in the sum because it gives zero. Now back to the original equation:
\begin{equation}
0=\frac{1}{p}\sum_{r=1}^\infty q^{r-1}p-\sum_{s=1}^\infty sq^{s-1}p
\end{equation}
Now note that the first sum gives one and the second gives $\mu$ so
\begin{equation}
\mu=\frac{1}{p}
\end{equation}

We weren't asked to prove
\begin{equation}
\sum_{r=1}^\infty p(r)=1
\end{equation}
but in case you are interested it is discussed here
\begin{equation}
\sum_r q^{r-1}p=1
\end{equation}
For this we use the expansion
\begin{equation}
\frac{1}{1-q}=1+q+q^2+\ldots
\end{equation}
you can think of this as coming from the binomial expansion of $(1-q)^{-1}$ using the generalized binomial formula discovered by Newton
\begin{equation}
(1+x)^n=\sum_r \frac{n(n-1)\ldots (n-r+1)}{r!}x^r
\end{equation}
with $n=-1$; it can also be derived as the Taylor expansion. Either way we have
\begin{equation}
\frac{1}{1-q}=\sum_{r=0}^\infty q^r=\sum_{r=1}^\infty q^{r-1}
\end{equation}
and since $1-q=p$ this gives the result.

\end{enumerate}

\subsection*{Extra questions}


These are for you to do on your own, not for handing up. Solutions will be included in the solutions section. I haven't added these question yet, but they will be added to the online version of this problem sheet over the next couple of days.

\end{document}

