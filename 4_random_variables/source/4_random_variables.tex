%4_random_variables.tex
%notes for the course Probability and Statistics COMS10011
%taught at the University of Bristol
%2018_19 Conor Houghton conor.houghton@bristol.ac.uk
%To the extent possible under law, the author has dedicated all copyright
%and related and neighboring rights to these notes to the public domain 
%worldwide. These notes are distributed without any warranty. 

\documentclass[11pt,a4paper]{scrartcl}
\typearea{12}
\usepackage{graphicx}
%\usepackage{pstricks}
\usepackage{listings}
\usepackage{color}
\usepackage{slashbox}
\lstset{language=C}
\usepackage{fancyhdr}
\pagestyle{fancy}
\lfoot{\texttt{github.com/COMS10011/2018\_19}}
\lhead{COMS100011 4\_random\_variables - Conor}
\begin{document}

\section*{4 Random variables}

Very frequently when we are applying statistics and probability we are
dealing with data we want to quantify. Because we can add, subtract
and multiply numerical data, applying probability to numerical data allows for some
techniques and approaches that aren't possible for other data
types. The mathematical construction used in this case is a
\textsl{random variable}.

The expression \lq{}random variable\rq{} is often used very loosely
and in a way that can be confusing; however, it does have an exact
meaning. A random variable $X$ is a map from a sample space to a set
of numerical values. For now we will deal with a discrete random
variables, where those values are discrete. As an example consider the
sample space for flipping a coin three times; the sample space will
have eight elements, \{HHH, HHT, HTH, THH, HTT, THT, TTH, TTT\}. One
random variable, say $X$, on this space is the number of heads. This
random variable maps HHH to three and TTH to one.

Just like events have a probability, so do random variables, in fact
it is nearly the same definition: the probability that $X=x$,
$p(X=x)$, sometimes written $p(x)$, is the sum of the probabilities of
all the outcomes with value $x$. For a sample space $\mathcal{S}$ with
random variable $X$ it is the probability of the event:
\begin{equation}
E_x=\{s\in \mathcal{S}|X(s)=x\}
\end{equation}
When there are more than one random variable being considered we
sometimes write $p_X(x)$ to mean the probability corresponding to the
random variable $X$. Note that we use a small letter $p$ for
probabilities when we are talking about random variables, a big letter
$P$ is usually used when talking about events.

Going back to the coin flipping example, it is easy to see that 
\begin{equation}
p(X=1)=\frac{3}{8}
\end{equation}
since there are three equally probable outcomes corresponding to
$X=1$. We can put all the possible values of the random variable in to
a table
\begin{center}
\begin{tabular}{c|cccc}
&0&1&2&3\\
\hline
$p$&1/8&3/8&3/8&1/8
\end{tabular}
\end{center}
A table like this is called a \textsl{probability distribution}.

The probabilities satisfy similar properties to the probabilities for
events:
\begin{equation}
0\le p(x)\le 1
\end{equation}
and 
\begin{equation}
\sum_x{p(x)}=1
\end{equation}
where the sum is over all possible values $x$ with non-zero
probabilities. More importantly, we interpret $p(x)$ as the frequency
$X=x$, so if we choose random elements of the sample space 
\begin{equation}
\frac{\mbox{number of times we get the value }x}{\mbox{number of samples we take}}\rightarrow p_X(x)
\end{equation}
with the left hand side approaching the right hand side as the number
of samples goes to infinity.

As another example, consider rolling two three-sided dice, I'll leave it to you to work
out how to make a three-sided dice, having a more realistic number of
dice faces makes the tables really big. Let $X$ be the sum of the two
numbers and $Y$ be the larger of the two numbers. We can work out both
distributions. For $X$ we have
\begin{center}
\begin{tabular}{c|ccccc}
&2&3&4&5&6\\
\hline
$p_X$&1/9&2/9&1/3&2/9&1/9
\end{tabular}
\end{center}
and for $Y$
\begin{center}
\begin{tabular}{c|ccc}
&1&2&3\\
\hline
$p_Y$&1/9&1/3&5/9
\end{tabular}
\end{center}
We won't discuss it much now but you could also work out a \textsl{joint distribution}, this is the
distribution for pairs $(X,Y)$:
\begin{center}
\begin{tabular}{c|ccccc}
\backslashbox{$Y$}{$X$}&2&3&4&5&6\\
\hline
1&  1/9&0&0&0&0\\
2&  0&2/9&1/9&0&0\cr
3&  0&0&2/9&2/9&1/9
\end{tabular}
\end{center}
so $p_{X,Y}(3,2)=2/9$ since it corresponds to the roll one and three
along with the roll three and one. We can recover the original
\textsl{marginal} distributions $p_X$ and $p_Y$ by summing out the
variable we aren't interested in; clearly
\begin{equation}
p_Y(y=2)=p_{X,Y}(2,2)+p_{X,Y}(3,2)+p_{X,Y}(4,2)+p_{X,Y}(5,2)+p_{X,Y}(6,2)=\frac{2}{9}+\frac{1}{9}=\frac{1}{3}
\end{equation}
We can also work out, again with the obvious notation, \textsl{conditional probabilities} $p_{X|Y}(x|y)$ so
\begin{center}
\begin{tabular}{c|ccccc}
&2&3&4&5&6\\
\hline
$p_{X|Y}(x|Y=2)$&0&2/3&1/3&0&0
\end{tabular}
\end{center}
which is just
\begin{equation}
p_{X|Y}(x|y)=\frac{p_{X,Y}(x,y)}{p_Y(y)}
\end{equation}

\subsection*{Expected values}

So far we have introduced some new, and slightly confusing, notation
without anything to show for it. However, there is an advantage; as we
noted above, numerical values support arithmetic. This allows us to
define the expectation value. If $g(x)$ is a function we define the
\textsl{expected value} of $g(X)$ as
\begin{equation}
\langle g(X)\rangle = \sum_x p(x)g(x)
\end{equation}
where the sum is over all values $x$ with non-zero probability. The
angle brackets $\langle \ldots \rangle$ are a common notation for
expectation values; they lead to a very convenient notation called
\lq{}bra and ket\rq{} notation; we won't get to that here but we will
use angle brackets for expectation values. Another common notation is to use a capital $E$, as in
\begin{equation}
E[g(X)]=\langle g(X)\rangle
\end{equation}

The most obvious expected value is the \textsl{expected value} of $X$:
\begin{equation}
\langle X \rangle =  \sum_x x p(x)
\end{equation}
Clearly, if the $p(x)$ represent the frequencies, this is the average
or \textsl{mean} value taken by $X$; it is often called $\mu$. We are
being careful to distinguish the expected value of $X$ and the mean;
there are occasions when this distinction matters, but generally it
doesn't. Basically the expected value is defined no matter what the
$p(x)$s represent, if, as they almost always do, they represent the
frequencies, what we think of as the probability of getting $X=x$,
then the expected value is equal to the mean. On the way to proving
some theorem it might be useful to define another probability
distribution on the same data, where the second distribution doesn't
correspond to frequencies; this is why it can be useful to distinguish
the mean and the expected value of $X$.

The expect value of $X$ is also referred to as the \textsl{first
  moment}; the \lq{}first\rq{} bit is because it is the expectation
value for the first power of $X$.

As a simple example consider the probability distribution above for
the number of heads when a coin is flipped three times:
\begin{equation}
\langle X\rangle =0\times\frac{1}{8}+1\times\frac{3}{8}+2\times\frac{3}{8}+3\times \frac{1}{8}=\frac{3+6+3}{8}=\frac{3}{2}
\end{equation}

Another common expected value is the \textsl{variance}:
\begin{equation}
V(X)=\langle (X-\mu)^2\rangle
\end{equation}
If $p(x)$ represents the frequencies, this is the square of the
standard deviation: $V(X)=\sigma^2$. It is often interpreted as measuring how spread out the data is. For example, this distribution:
\begin{center}
\begin{tabular}{c|cccc}
&0&1&2&3\\
\hline
$p_X$&1/8&3/8&3/8&1/8
\end{tabular}
\end{center}
and this distribution
\begin{center}
\begin{tabular}{c|cccc}
&0&1&2&3\\
\hline
$p_Y$&1/16&7/16&7/16&1/16
\end{tabular}
\end{center}
both have expect valued $1.5$:
\begin{equation}
\langle X\rangle = \langle Y\rangle = 1.5
\end{equation}
However,
\begin{equation}
V(X)=0.75
\end{equation}
whereas 
\begin{equation}
V(Y)=0.5
\end{equation}
reflecting the way the second distribution is more tightly bunched up
around the mean.

The expected value has nice properties that it inherits from the nice
properties of the underlying arithmetic. First, scalar multiplication:
for a constant $c$
\begin{equation}
\langle c g(X)\rangle =\sum_x cg(x)p(x)=c\sum_x g(x)p(x)=c\langle g(X)\rangle
\end{equation}
Also, trivially:
\begin{equation}
\langle 1\rangle=\sum_x p(x)=1
\end{equation}
Finally, it is additive: if $g_1$ and $g_2$ are two different
functions
\begin{eqnarray}
\langle (g_1(X)+g_2(X)\rangle &=&\sum_x [g_1(x)+g_2(x)p(x)\cr
&&=\sum_x g_1(x)p(x)+\sum_x g_2(x)p(x)=\langle g_1(X)\rangle+\langle g_2(X)\rangle
\end{eqnarray}

We can use this to get another formula for the variance:
\begin{equation}
V(X)=\langle (X-\mu)^2\rangle=\langle (X^2-2\mu X+\mu^2)\rangle
\end{equation}
Now, using the additive property
\begin{equation}
V(X)=\langle X^2 \rangle -2\mu\langle X\rangle +\langle \mu^2 \rangle
\end{equation}
Finally, noting $\mu=\langle X\rangle$ and using the scalar multiplication property, this give
\begin{equation}
V(X)=\langle X^2 \rangle - \mu^2 \rangle
\end{equation}
Incidentally $\langle X^2\rangle$ is called the \textsl{second
  moment}, the variance is called the \textsl{second central moment};
the \lq{}central\rq{} indicates that it is the second moment you get
if you take away the mean first.

\end{document}




