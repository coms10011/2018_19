\ifind
\section*{Summary}
\else
\subsection*{10 Central Limit Theorem}
\fi

\begin{itemize}
\item If $X$ and $Y$ are continuous random variables, with density
functions $p_X(x)$ and $p_Y(y)$ and 
\begin{equation}
Z=X+Y
\end{equation}
then
\begin{equation}
p_Z(z)=\int_{-\infty}^\infty {p_X(x)p_Y(z-x)dx}
\end{equation}
This calculation is called a \textbf{convolution}.

\item If  $X\sim\mathcal{N}(\mu_x,\sigma_x^2)$ and $Y\sim\mathcal{N}(\mu_Y,\sigma_Y^2)$ then
  \begin{equation}
X+Y=Z\sim \mathcal{N}(\mu_X+\mu_Y,\sigma_X^2+\sigma_Y^2)
  \end{equation}

\item Let $\{X_1,X_2,\ldots,X_n\}$ be a set of random variables. A set of
random variables is called i\textbf{ndependent identically distributed},
usually abbreviated to i.i.d., if the variables all have the same
probability density, say $p_X(x)$ and are independent.
  
\item The \textbf{Central limit theorem}: if  $\{X_1,X_2,\ldots,X_n\}$ is i.i,d, the \textbf{sample mean} is
  \begin{equation}
S_n=\frac{X_1+X_2+\ldots+X_n}{n}
\end{equation}
  As $n$ approaches infinity
\begin{equation}
U_n=\sqrt{n}\left(\frac{S_n-\mu}{\sigma}\right) \sim \mathcal{N}(0,1)
\end{equation}
\end{itemize}
