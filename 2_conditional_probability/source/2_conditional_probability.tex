%1_probability.tex
%notes for the course Probability and Statistics COMS10011 
%taught at the University of Bristol
%2018_19 Conor Houghton conor.houghton@bristol.ac.uk

%To the extent possible under law, the author has dedicated all copyright 
%and related and neighboring rights to these notes to the public domain 
%worldwide. These notes are distributed without any warranty. 

\documentclass[11pt,a4paper]{scrartcl}
\typearea{12}
\usepackage{graphicx}
%\usepackage{pstricks}
\usepackage{listings}
\usepackage{color}
\lstset{language=C}
\usepackage{fancyhdr}
\pagestyle{fancy}
\lfoot{\texttt{github.com/COMS10011/2018\_19}}
\lhead{COMS100011 2\_conditional\_probability - Conor}
\begin{document}

\section*{Conditional Probability}

One of the main characters in the Ian Banks novel \textif{Canal
  Dreams} is afraid of flying; people try to reason her out of it,
pointing out that being in a plane crash is very unlikely. She points
out in return that it is even less likely if you aren't on a plane. We
often want to model random or unknown events that are related to each
other; for example, on a rare dry day in Summer in Galway, where I
grew up, it was more likely to rain the next day if you could see the
hills of Claire clearly across the bay. This sort of situation is
modelled using conditional probabilities.

Consider the sample space made up of pairs of points:
\begin{equation}
X=\{(r,c),(r,u),(d,c),(d,u)\}
\end{equation}
The outcome $(r,c)$ corresponds to rain and a clear view of the hills,
$(r,u)$ to rain and no clear view, $(d,c)$ to dry and a clear view and
$(d,u)$ to dry and no clear view. Let's imagine we have probabilities
for each of these outcomes:
\begin{center}
\begin{tabular}{c|cccc}
&$(r,c)$&$(r,u)$&$(d,c)$&$(d,u)$\\
p&1/2&1/4&1/16&3/16
\end{tabular}
\end{center}
Thus, the probability that it is going to rain and the hills are clear
is $1/2$; the probability that the hills are unclear and it is going
to stay dry is $3/16$. 

It is useful to also define some events, for example $R$ for rain
\begin{equation}
R=\{(r,c),(r,u)\}
\end{equation}
and $C$ for clear
\begin{equation}
C=\{(r,c),(d,c)\}
\end{equation}
Equally well $\bar{R}=\{(d,c),(d,u)\}$ is the set of all points not in $R$, it is called the \textbf{complement} of $R$ and is given by
\begin{equation}
\bar{R}=X\setminus R
\end{equation}
where \lq{}$\setminus$\rq{} is \textbf{set minus}. It is easy to work out
\begin{equation}
P(R)=3/4
\end{equation}
and 
\begin{equation}
P(C)=11/16
\end{equation}
Now we can also work out probabilities for the intersections, for example:
\begin{equation}
P(R\cap C)=1/2
\end{equation}
This is the probability of $R$ and $C$ both happening, in the small
example we are looking at here, this is just one outcome $(r,c)$, but,
in general, the intersection might have more points in it. Now, notice
that by definition
\begin{equation}
C=(R\cap C)\cup (\bar{R}\cap C)
\end{equation}
and, since $(R\cap C)\cap (\bar{R}\cap C)=\emptyset$ then
\begin{equation}
P(C)=P(R\cap C)+P(\bar{R}\cap C)
\end{equation}
Hence, the probability of $C$ is the probability of $C$ and $R$ plus
the probability of $C$ and not $R$.



\end{document}

