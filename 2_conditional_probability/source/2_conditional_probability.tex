%1_probability.tex
%notes for the course Probability and Statistics COMS10011 
%taught at the University of Bristol
%2018_19 Conor Houghton conor.houghton@bristol.ac.uk

%To the extent possible under law, the author has dedicated all copyright 
%and related and neighboring rights to these notes to the public domain 
%worldwide. These notes are distributed without any warranty. 

\documentclass[11pt,a4paper]{scrartcl}
\typearea{12}
\usepackage{graphicx}
%\usepackage{pstricks}
\usepackage{listings}
\usepackage{color}
\lstset{language=C}
\usepackage{fancyhdr}
\pagestyle{fancy}
\lfoot{\texttt{github.com/COMS10011/2018\_19}}
\lhead{COMS100011 2\_conditional\_probability - Conor}
\begin{document}

\section*{The Monty Hall Problem}

The Monty Hall problem has become too much of a classic, it is taught
so often students sometimes get tired of hearing about it. It became
well known in 1990 when Marilyn von Savant, an author who had a puzzle
column in Parade magazine, wrote about the problem. Thousands of irate
and misguided men wrote to incorrectly point out errors in what she
had written, as such it The Monty Hall problem teaches us about the
behaviour of irate men as well as statistics. It even features in Ian
McEwan's novel Sweet Tooth. As such we won't look at the solution to
the problem, instead I will describe the problem and you can work it
out yourselves, one way to approach the solution is to use conditional
probability, the subject of this set of notes.

The Monty Hall problem is based on a game show \lq{}Let's make a
deal\rq{}, hosted by Monty Hall, who died at the age of 96 in 2017. In
the problem, a kind of idealization of what really went on in the
show, a contestant who does well gets the chance to try for the big
prize, a car. They are presented with three doors and are asked to
choose one: the car is behind one door, behind the other two are
goats. The contestant, foolishly in my view, wants the car. They pick
a door. Monty Hall then opens one of the two remaining doors to reveal
a goat. He gives the contestant the chance to change doors. Should
they change?

\section*{Conditional Probability}

One of the main characters in the Ian Banks novel \textit{Canal
  Dreams} is afraid of flying; people try to reason her out of it,
pointing out that being in a plane crash is very unlikely. She points
out in return that it is even less likely if you aren't on a plane. We
often want to model random or unknown events that are related to each
other; for example, on a rare dry day in Summer in Galway, where I
grew up, it was more likely to rain the next day if you could see the
hills of Clare clearly across the bay. This sort of situation is
modelled using conditional probabilities.

Consider the sample space made up of pairs of points:
\begin{equation}
X=\{(r,c),(r,u),(d,c),(d,u)\}
\end{equation}
The outcome $(r,c)$ corresponds to rain and a clear view of the hills,
$(r,u)$ to rain and no clear view, $(d,c)$ to dry and a clear view and
$(d,u)$ to dry and no clear view. Let's imagine we have probabilities
for each of these outcomes:
\begin{center}
\begin{tabular}{c|cccc}
&$(r,c)$&$(r,u)$&$(d,c)$&$(d,u)$\\
p&1/2&1/4&1/16&3/16
\end{tabular}
\end{center}
Thus, the probability that it is going to rain and the hills are clear
is $1/2$; the probability that the hills are unclear and it is going
to stay dry is $3/16$. 

It is useful to also define some events, for example $R$ for rain
\begin{equation}
R=\{(r,c),(r,u)\}
\end{equation}
and $C$ for clear
\begin{equation}
C=\{(r,c),(d,c)\}
\end{equation}
Equally well $\bar{R}=\{(d,c),(d,u)\}$ is the set of all points not in $R$, it is called the \textbf{complement} of $R$ and is given by
\begin{equation}
\bar{R}=X\setminus R
\end{equation}
where \lq{}$\setminus$\rq{} is \textbf{set minus}. This event
corresponds to a dry day. It is easy to work out
\begin{equation}
P(R)=3/4
\end{equation}
and 
\begin{equation}
P(C)=11/16
\end{equation}
Now we can also work out probabilities for the intersections, for example:
\begin{equation}
P(R\cap C)=1/2
\end{equation}
This is the probability of $R$ and $C$ both happening, in the small
example we are looking at here, this is just one outcome $(r,c)$, but,
in general, the intersection might have more points in it. Now, notice
that by definition
\begin{equation}
C=(R\cap C)\cup (\bar{R}\cap C)
\end{equation}
and, since $(R\cap C)\cap (\bar{R}\cap C)=\emptyset$ then
\begin{equation}
P(C)=P(R\cap C)+P(\bar{R}\cap C)
\end{equation}
Hence, the probability of $C$ is the probability of $C$ and $R$ plus
the probability of $C$ and not $R$, or, in words; the probability that
it is clear is the probability that it is clear and raining plus the
probability that it is clear and not raining.

Now we define
\begin{equation}
P(R|C)=\frac{P(R\cap C)}{P(C)} 
\end{equation}
This is the probability of $R$ given $C$; if we know it is clear how
likely is it to be raining. The demoninator $P(C)$ is needed because
the probabilities must add up to one and $P(C)=P(R\cap C)+P(\bar{R}\cap C)$. Thus
\begin{equation}
P(R|C)=\frac{1/2}{11/16}=\frac{8}{11}\approx  0.73
\end{equation}
In other words, if we can see the hills of Clare, there is a 73\%
chance it will rain tomorrow.

More generally, if we have two events $A$ and $B$ then the conditional probability is
\begin{equation}
P(A|B)=\frac{P(A\cap B)}{P(B)}
\end{equation}

Here is a quick example. In the card game 21 you get given two cards
and you add up there value; you then decide whether to take more
cards, with the aim of being no more than and as close to 21 as
possible. In 21 picture cards count as ten; you can count an ace as
one or 11; for simplicity lets pretend an ace has to be one and we are
concentrating on the original hand of two cards. Now imagine you are
given two cards; if the first card is a king what is the chance your
hand is bigger the 17 in value? Let $B$ be the set of hands that has a
king as the first card and $A$ is the set of hands worth more than
17. First lets calculate $P(B)$. Well there are $52\times 51$ hands in
total, and $4\times 51$ of them have a king as the first card so
\begin{equation}
P(B)=\frac{4}{52}=\frac{1}{13}
\end{equation}
Now, lets count the number of hands with a king as the first card, and
a total value over 17. If the first hand is a king, if the second card
is an 8, 9, 10 or picture card. This might seem like there are
$6\times 4=24$ cards to go along with the king, but it isn't, since
the first card is a king there are only 23 other cards to go along
with it to make 18 or higher. Thus
\begin{equation}
P(A\cap B)=\frac{4\times 23}{51\times 52}
\end{equation}
where the first four corresponds to the choice of king. Thus
\begin{equation}
P(A|B)=\frac{23}{51}\approx 0.45
\end{equation}

In fact, conditional probability is very frequently calculated in
situations like the first one, where the sample space is made of
pairs. This make sense, it is like there are two things to measure and
you use conditional probability to see how one relates to the
other. Here is another example; imagine thinking about the journey
times from Temple Meades to the top of Park Street. Consider a table
of the describing the frequency different modes of transport are used and different times of travel are measured.
\begin{center}
\begin{tabular}{c|ccc}
        &car    &walk  &cycle\\
\hline
$<20$     &0.025  &0.1   &0.2\\
$[20,40]$ &0.125  &0.35  &0.05\\
$>40$     &0.1    &0.05  &0
\end{tabular}
\end{center}
Using the obvious notation we can work out the \textbf{marginal}
distributions: for vehicles $P(car)=P(cycle)=0.25$ wherese
$P(walk)=0.5$, for times $P(<20)=0.325$, $P([20,40])=0.525$ and
$P(>40)=0.15$. Now, say someone rushes in to your meeting and says
they are sorry but it took them over 40 minutes to get from Temple
Meades, what is the chance they travelled by car:
\begin{equation}
P(\mbox{car}\,|>40)=\frac{P(\mbox{car}\,\cap >40)}{P(>40)}=\frac{0.1}{0.15}=\frac{2}{3}
\end{equation}



\end{document}

