%1_probability.tex
%notes for the course Probability and Statistics COMS10011 
%taught at the University of Bristol
%2018_19 Conor Houghton conor.houghton@bristol.ac.uk

%To the extent possible under law, the author has dedicated all copyright 
%and related and neighboring rights to these notes to the public domain 
%worldwide. These notes are distributed without any warranty. 

\documentclass[11pt,a4paper]{scrartcl}
\typearea{12}
\usepackage{graphicx}
%\usepackage{pstricks}
\usepackage{listings}
\usepackage{color}
\lstset{language=C}
\usepackage{fancyhdr}
\pagestyle{fancy}
\lfoot{\texttt{github.com/COMS10011/2018\_19}}
\lhead{COMS10007 1\_probability - Conor}
\begin{document}

\section*{Introduction}

In probability and statistics we study develop the tools to develop
the tools for studying data that is uncertain or noisy and processes
that have a random element. Probability and statistics is important to
computer scientists of all types; some computer scientists use
computers to manipulate data, or to make statistical inferences about
data, for example, in machine learning, other computer scientists make
products that need to be tested and statistics is needed to interpret
the results of user testing.

Imaging a gambler wants to decide if a coin is fair; imagine they toss
it 20 times and get a head each time. Can they decide that the coin is
fair. Immeadiately it is clear that they can't say that it is
impossible that a fair coin would turn up heads five times; it is just
pretty unlikely. They could say that there is only a one in
\begin{equation}
2^{5}=32
\end{equation}
chance that a fair coin would produce that result. Since
$(1-1/32)*100\approx 97$ it would seem that gambler could be $97\%$
certain the coin was unfair. This isn't the end of the story though:
five harps would have been equally suprising; so perhaps the thing to
say is that there is only a one in
\begin{equation}
2^{4}=16
\end{equation}
the coin would produce such an unlikely result. Does this means the
gambler can only be $94\%$ certain the coin is unfair? The idea of
this course it to learn how to calculate probabilities, the
probability of five heads is an easy example of this, and how to make
inferences, such as deciding the chance the coin is unfair.


%% change to 1 9 90 900 etc
It is very easy to confuse yourself with probability; here is a
confusing problem. An evil genius has decided to tattoo their names on
some peoples foreheads and in the usual over-elaborate evil genius
manner they do it according to a diabolic game. A person is selected
and forced into the tattoo parlour, the evil genius rolls two die and
if they show two sixes, they tatoo the person and the game ends,
otherwise the lucky person is released and the evil genius selects two
new people. Again, the evil genius rolls the die, again, if gives two
sixes then they tattoo everyone and the game ends, if not, they let
both leave and select four people and roll the die. The process
repeats, doubling the number each time, until the dice comes up with
two sixes, the people in the parlour are tattooed and the game ends. If you are
forced into the parlous, what is the chance you'll end up with a tattoo
on your forehead. Is it the relatively modest one in 36, or, given
that a majority of people who enter the room end up tatooed, is it
nearer one in two?

To avoid getting confused we need some notation and some mathematical
machinery and that's what we are going to do first.

\section*{1: Probability}

To start with we need to terminology and notation and this begins with
a sample space $\mathcal{X}$, this is the set of points; the idea is
that these points are the possible outcomes of an
experiment. Initially it is useful to think of discrete sample spaces,
for example, for the coin, the sample space has two points:
\begin{equation}
\mathcal{S}=\{H,T\}
\end{equation}
where the two points $H$ and $T$ correspond to heads and harps, the
two possible outcomes of flipping the coin. Later we will look at
continuous sample spaces where the experiment takes a value in a
continuum, so we could have
\begin{equation}
\mathcal{L}=[0,\infty)
\end{equation}
is the length of a song on the radio.







\subsection*{Introduction}

\end{document}

